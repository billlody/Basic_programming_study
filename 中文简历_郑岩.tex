%!Mode:: "TeX:UTF-8"
% yzheng's Resume
% Created: 09 Mar 2011
% Last Modified: 11 July 2011

\documentclass[a4paper,9pt,oneside]{scrartcl}
\usepackage{geometry}
\usepackage[T1]{fontenc}
\usepackage{CJKutf8}

\pagestyle{empty}
\geometry{letterpaper,tmargin=0.4in,bmargin=0in,lmargin=0.8in,rmargin=0.8in,headheight=0in,headsep=0in,footskip=.3in}

\setlength{\parindent}{0in} \setlength{\parskip}{-1pt}
\setlength{\itemsep}{-1pt} \setlength{\topsep}{-1pt}
\setlength{\partopsep}{-1pt} \setlength{\tabcolsep}{0in}
%\renewcommand{\baselinestretch}{0.5}
%\setlength{\parskip}{0em}


% Name and contact information
\newcommand{\name}{Zheng Yan}
\newcommand{\addr}{Huayanbeili Road, Chaoyang District, Beijing, China}
\newcommand{\hkphone}{HK Tel:(+852)51652919}
\newcommand{\bjphone}{BJ Tel:(+86)13671288621}
\newcommand{\email}{Email:billzhengyan@hotmail.com}
\newcommand{\Region}{Hong Kong, China}

%%%%%%%%%%%%%%%%%%%%%%%%%%%%%%%%%%%%%%%%%%%%%%%%%%%%%%%%%
% New commands and environments

% This defines how the name looks
\newcommand{\bigname}[1]{
    \begin{center}\fontfamily{phv}\selectfont\LARGE\bfseries#1\end{center}
}

% A ressection is a main section (<H1>Section</H1>)
\newenvironment{ressection}[1]{
    \vspace{2pt}
    {\large#1}
    \begin{itemize}
    \vspace{2pt}
}{
    \end{itemize}
}

% A resitem is a simple list element in a ressection (first level)
\newcommand{\resitem}[1]{
    \vspace{-4pt}
    \item #1
}

\newcommand{\resitems}[1]{
    \vspace{-4pt}
    \item #1
%    \itshape #2
}

% A ressubitem is a simple list element in anything but a ressection (second level)
\newcommand{\ressubitem}[1]{
    \vspace{0pt}
    \item #1
}

% A resbigitem is a complex list element for stuff like jobs and education:
%  Arg 1: Name of company or university
%  Arg 2: Location
%  Arg 3: Title and/or date range
\newcommand{\resbigitem}[3]{
    \vspace{-5pt}
    \item
    \textbf{#1}\\
    #2
    \textit{#3}
}

% This is a list that comes with a resbigitem
\newenvironment{ressubsec}[3]{
\resbigitem{#1}{#2}{#3}
    \vspace{-2pt}
    \begin{itemize}
}{
    \end{itemize}
}

% This is a simple sublist
\newenvironment{reslist}[1]{
    \resitem{\textbf{#1}}
    \vspace{-5pt}
    \begin{itemize}
}{
    \end{itemize}
}



%%%%%%%%%%%%%%%%%%%%%%%%%%%%%%%%%%%%%%%%%%%%%%%%%%%%%%%%%
% Now for the actual document:

\begin{document}

\fontfamily{ppl} \selectfont

\begin{CJK*}{UTF8}{gbsn}
% Name with horizontal rule
\bigname{\Huge 郑岩}

\vspace{-6pt} \rule{\textwidth}{1pt}

\vspace{-1pt} {\small\itshape 北京市朝阳区 \hfill 电话:+86 13671288621\\
       Email:billzhengyan@hotmail.com  %\hfill 北京电话:86 13671288621
%\Region
%\hfill Expected Salary: 15,000 HKD/mon
%\flushright Studentship: 13,000 HKD/mon
\\[5pt]}
具备较好的数理建模能力和机器学习在实际业务中的经验,同时具有较强的c++/python编程工程能力,希望应用最新技术于实际问题
\\[-6mm]

\rule{\textwidth}{1pt}

\vspace{6pt}

%%%%%%%%%%%%%%%%%%%%%%%%
\begin{ressection}{经历:}
  \begin{ressubsec}{瑞士再保险 \hfill 北京}{高级数据科学家}{\hfill 2019年8月— — 现今}
	  \ressubitem{生存分析建模(surrogate model)、数据挖掘(association rule)、可解释机器学习(Shapley)的归因分析}
	  \ressubitem{重疾险预测性核保(Deep learning、xgboost)、NLP疾病信息检测系统(文本摘要算法)}
	  \ressubitem{其余数据分析相关开发工作}
  \end{ressubsec}
  \begin{ressubsec}{蚂蚁金服 \hfill 北京}{高级算法工程师}{\hfill 2017年5月 -- 2019年8月}
	  \ressubitem{C++代码搭建完整快速回测平台}
	  \ressubitem{机器学习(LASSO, GBDT)、深度学习构建多因子股票资产组合}
	  \ressubitem{微贷花呗流动性业务预测:采用CNN, Adanet, GBDT, 迁移学习等模型,预测花呗流动性}
	  \ressubitem{针对股票特征工程问题:完善遗传算法系统;采用强化学习研究遗传算法中的公式解析(TRPO, PPO)}
	  \ressubitem{股票另类数据挖掘:采用LTR(RankNet)模型在股票主营业务层面进行因子建模, 对股票知识图谱进行图模型(MPNN)探索}
  \end{ressubsec}
  \begin{ressubsec}{九坤投资 \hfill 北京}{量化策略研究员}{\hfill 2015年12月 -- 2017年4月}
    \ressubitem{量化研究:研究Alpha策略、参与开发多维数据(基本面、LVL2、分钟线、现金流、价量等)GA系统}
    \ressubitem{分析Level2数据构建资金流向,采用VPIN等指标产生基于此的alpha策略}
    \ressubitem{研究实践若干金融理论因子,诸如Copula函数、因子聚类}
	\ressubitem{研究股票日线择时策略}
  \end{ressubsec}
  \begin{ressubsec}{香港科技大学物理系 \hfill 香港}{研究项目:
    铁基超导体、二维自旋体系比热测量与研究}{\hfill  2009年12月 -- 2014年9月}
    \ressubitem{研究工作:高压下采用 AC-Modelated 方法测量 122 体系铁基超导体、SCBO
        材料低温下的比热、电学特性;编写测量程序控制仪器参数(C语言为主的Labwindows软件与GPIB接口联通)}
    \ressubitem{后期工作:整理分析数据;用\LaTeX完成论文;用英语通过答辩}
  \end{ressubsec}
    %% \begin{ressubsec}{香港科技大学物理系}{助教}{\hfill 2009年09月--2012年06月}
    %%     \ressubitem{实验助教:用英语指导学生实验,评定学生水平,分别帮助不同的学生}
    %%     \ressubitem{授课助教:用英语授课,答疑,判学生作业及考试卷}
    %% \end{ressubsec}
\end{ressection}

%%%%%%%%%%%%%%%%%%%%%%%%

%\begin{ressection}{金融知识:}
%    \resitems{ 基金从业资格}
    %\resitems{{复旦大学物理系 人民奖学金} \hfill \emph{2005--2008}}
    %\resitems{{中学生物理竞赛北京赛区一等奖、二等奖} \hfill \emph{2003--2004}~~}
%\end{ressection}


%%%%%%%%%%%%%%%%%%%%%%%%
\begin{ressection}{教育背景:}
    \begin{ressubsec}{香港科技大学}{硕士, 博士} {专业:物理} 
        \ressubitem{\itshape 时间:2009年9月-- 2015年6月}
    \end{ressubsec}
    \begin{ressubsec}{复旦大学}{本科}{专业:物理}
        \ressubitem{\itshape 时间:2005年9月 -- 2009年6月}
    \end{ressubsec}
\end{ressection}


\begin{ressection}{发表文章:}
	\resitems{\textbf{Y. Zheng}, Y. Wang, B. Lv, C. W. Chu, and R. Lortz, \emph{New Journal of Physics} \textbf{14}, 053034 (2012)}
	\resitems{\textbf{Y. Zheng}, Y. Wang, F. Hardy, T. Wolf, C. Meingast, and R. Lortz, \emph{Phys. Rev. B} \textbf{89}, 054514 (2014)}
	\resitems{\textbf{Zheng. Yan}, Liu. Y, Toyota. Naoki, Lortz. Rolf, \emph{Journal of Physics: Condensed Matter}, \textbf{27} 075701 (2015)}
	\resitems{\textbf{Zheng. Yan}, Pok Man Tam, Jianqiang Hou, Anna E. BÃ˝uhmer, Thomas Wolf, Christoph Meingast, Lortz. Rolf, \emph{Phys. Rev. B} \textbf{93}, 104516
	(2015)}
\end{ressection}

%\begin{ressection}{金融知识:}
%    \resitems{ 基金从业资格}
    %\resitems{{复旦大学物理系 人民奖学金} \hfill \emph{2005--2008}}
    %\resitems{{中学生物理竞赛北京赛区一等奖、二等奖} \hfill \emph{2003--2004}~~}
%\end{ressection}

%%%%%%%%%%%%%%%%%%%%%%%%
%\begin{ressection}{其他:}
%    \resitems{\textbf{爱好:} 钢琴, 游泳}
%\end{ressection}

%%%%%%%%%%%%%%%%%%%%%%%%
\begin{ressection}{技能:}
    \resitem{\textbf{计算机技能:}  熟悉Python, C/C++, tensorflow(Linux or Visual Studio), SQL; 之前有些经验在Erlang, Excel+VBA, Matlab, Mathematica \& Labview, 具有bash编程经验}
	\resitem{\textbf{机器学习:} 熟悉机器学习、深度学习}
    \resitem{\textbf{语言:} 英语:流利}

\end{ressection}

\vspace{2pt}
%%%%%%%%%%%%%%%%%%%%%%%%

\end{CJK*}
\end{document}
